\documentclass[./A14_Report.tex]{subfiles}

\begin{document}
\chapter{Conclusion}
\section{Results}
The following results were obtained on applying the compression algorithm on
images of various sizes. Note that $Max = \log_2{T_0}$.
\begin{table}[H]
\centering
\begin{tabular}{|c|c|c|c|}
    \hline
    \textbf{Image} & \textbf{Original size (bytes)} & \textbf{Size after
    compression (bytes)} & \textbf{Iterations}\\
    \hline
    eggs8 & 64 & 46 & 4\\
    eggs8 & 64 & 70 & 5\\
    eggs8 & 64 & 216 & Max\\
    \hline
    eggs16 & 256 & 153 & 6\\
    eggs16 & 256 & 257 & 7\\
    eggs16 & 256 & 621 & Max\\
    \hline
    lichten & 262144 & 198751 & 14\\
    lichten & 262144 & 269263 & 15\\
    lichten & 262144 & 352671 & Max\\
    \hline
\end{tabular}
\caption{Data collected}
\label{tab:data}
\end{table}

A key observation that can be made is that the algorithm produces files with
lower sizes than the originals only up to a certain maximum, after which the
algorithm generates files \textit{larger} than the original files. This is due
to redundancy that results from a very low threshold. Barring this, the
algorithm shows promising results for compression with minimum loss of data.

\section{Remarks}
Undertaking this project has enabled us to explore the extensive research
conducted in the field of \textit{image compression}. We established an
understanding of the \textit{Discrete Wavelet Transform} and it's
immediate application in the Embedded Zerotree Wavelet Algorithm. We also
discovered the various aspects such as encoding schemes and quantization
methods that govern the EZW algorithm output.

\section{Future work}
\begin{itemize}
        \item Write the de-compressor
        \item Include entropy encoder
        \item A multi-core implementation of the algorithm
        \item An FPGA-based hardware accelerator for the algorithm
\end{itemize}
\end{document}
